 \input ctustyle
 \input glosdata
 \input opmac-bib
 \worktype [B/CZ]
 \faculty {F3}
 \department {Katedra řídicí techniky}
 \title {Řízení bezkartáčových motorů s deskou Raspberry Pi a Linuxem}
 \author {Martin Prudek}
 \date {Duben 2015}
 \abstractEN {Testing}
 \abstractCZ {Testovani}
 \declaration {Prohlasuji, ze jsem pracoval poctive.}
 \makefront
 
 \def\frac#1#2{{\begingroup#1\endgroup\over#2}} %/frac je LaTex, definujme vlastni

 \chap Motivace
 
 Značné zlevnění a zpřístupnění jednodeskových minipočítačů v posledních letech u nás vyvolalo otázku, zda je možné některé hojně rozšířene modely využít k řídicím aplikacím.
 
  Z široké nabídky zařízení jsme vybrali Raspberry PI model B rev. 2.0. U tohoto počítače již byla prostudována možnost řízení motoru stejnosměrného motoru zachytáváním impulzů inkrementálního rotačního senzoru polohy\cite[Meciar]. Takové řízení je však vlivem nedostatečného výpočetního výkonu možné jen do 14kHz (2100ot/min). Proto bylo zpracování hrubých senzorových dat přesunuto do FPGA obvodu, se kterým Raspberry komunikuje prostřednictvím protokolu SPI.
  
  Jednosměrný bezkartáčový motor byl dále nahrazen efektivnějším PMS motorem.
 
 
 \chap PMS motory.

PMS (Permanent Magnet Synchronous) motory jsou díky vysoké efektivitě a robustní kontrukci bez kartáčů vhodnou volbou v mnoha řídicích aplikacích\cite[ESD]. 

Komutace probíhá na rozdíl od kartáčových motorů elektronicky, což přínáší vyšší požadavky na řídicí hardware. Odměnou je výšší výkon v poměru k váze, stejně tak točivý moment v poměru k příkonu. Výhodou jsou také nížší hlučnost a delší životnost, protože nedochází k opotřebení kartáčů a mechanických částí komutátoru. \cite[PMSM-Kinetis] Elektronická komutace bývá implementována v procesorovém systému, či speciálním obvodu (FPGA / ASIC)\cite[Meloun].
 
 \sec Matematický popis
 
 Uvažujme $i_{sa}$, $i_{sb}$ a $i_{sc}$ proudy procházející vinutím statoru, platí:
 
 $$ i_{sa} + i_{sb} + i_{sc} =0\eqmark $$
 
 Toto může být vyjádřeno jako vektor v komplexní rovině, potom:
 
 $$ \overline{i} = k(i_{sa}+ai_{sb}+a^2i_{sc}) \eqmark$$
 kde $a$ a $a^2$ jsou operátory posouvajíci fázi o 120°, $ a=e^{{2j\pi}/{3}} $ a $a^2=e^{{4j\pi}/{3}}$.
 $k$ je transformační konstanta.
 
 $\overline{i}$ můžeme vyjádřit jako součet jeho reálné a imaginární složky:

 $$ \overline{i}=i_{s\alpha}+ji_{s\beta} \eqmark$$
 
\medskip \clabel[proudy]{Komplexní vyjádření vektoru proudu vinutím statoru}
\picw=7cm \cinspic currents.png
\caption/f Komplexní vyjádření vektoru proudu vinutím statoru. $\alpha$ značí reálnou osu a $\beta$ značí imaginární osu.
\medskip %obrazek zustal tam, kde ma. \midinsert ho posunul

Se znalostí proudů, protékajících jednotlivými fázemi, pak můžeme vypočítat myšlené proudy tekoucí rovnoběžně s imaginární a reálnou osou.

 $$ 
 \eqalignno{ i_{s\alpha}&=k(i_{sa}-i_{sb}/2-i_{sc}/2) \cr
  i_{s\alpha}&=(2i_{sa}-i_{sb}-i_{sc})/3 & \eqmark \cr}
  $$
 
$$
\eqalignno{ i_{s\beta}&=k\sqrt{3}(i_{sb}-i_{sc})/2 \cr
	i_{s\beta}&=(i_{sb}-i_{sc})/\sqrt{3}  & \eqmark \cr}
$$

Pro volbu $k=2/3$.

Pro popis PMS motorů je uvažován ideálně symetrický motor se sinusoidně rozloženým vinutím. Pro takovou idealizaci uvažujeme napětí na vinutích $u_{sa}$, $u_{sb}$ a $u_{sc}$  následující:

$$
\label[rce1]
u_{sa}=R_si_{sa}+{{d}\over{dt}}\psi_{sa} \eqmark$$
$$ u_{sb}=R_si_{sb}+{{d}\over{dt}}\psi_{sb} \eqmark$$
$$ u_{sc}=R_si_{sc}+{{d}\over{dt}}\psi_{sc} \eqmark$$

kde $\psi_{sa}$,$\psi_{sb}$ a $\psi_{sc}$ jsou magnetické indukční toky vyvolané proudy odpovídajících vinutí. Vyjádření napětí vektorem v Gaussově(komplexní) rovině (Clarkova transformace) bude následující:

 $$ u_{s\alpha}=R_si_{s\alpha}+{{d}\over{dt}}\psi_{s\alpha} \eqmark$$
 $$ u_{s\beta}=R_si_{s\beta}+{{d}\over{dt}}\psi_{s\beta} \eqmark$$
 
 Přitom složky magnetického indukčního toku budou následující:
 
 $$\psi_{s\alpha} = L_{s\alpha}i_{s\alpha} + \psi_Mcos\theta_r \eqmark$$
 $$\psi_{s\beta} = L_{s\beta}i_{s\beta} + \psi_Msin\theta_r \eqmark$$
 
 kde $\theta_r$ je úhlová pozice rotoru a $\psi_M$ je magnetický indukční tok rotoru. $L_{s\alpha}$ a $L_{s\beta}$ jsou složky vzájemné indukčnosti rotor-stator.  

Úhlové zrychlení takového motoru o zátěži $T_L$ s $p$ póly připadajícími na každou fázi můžeme vyjádřit jako:

$$
{d\omega\over{dt}}=
{p\over{j}}\lbrack{3\over{2}}p(\psi_{s\alpha}i_{s\beta}-\psi_{s\beta}i_{s\alpha})-T_L\rbrack
$$

Rovnice \ref[rce1].

 




 
 
 
 
 \sec Konstrukce
 
 Text v sekci.
 
 \sec Řízení
 
 
 \chap Popis hardware
 
 \sec Použitý motor
 
 \sec Raspberry Pi
 
 Raspberry Pi je jednodeskový počítač založený na rodině architektury ARM, který se v současnosti dodavá v několika variantách.
 
 Základem první verze tohoto minipočítače je \glref{SoC} BCM2835, který obsahuje centrální procesor ARM1176JZF-S s taktem 700 MHz, grafický procesor VideoCore IV a 256 MB nebo 512 MB  paměti RAM. Neumožňuje však připojení pevného disku. Opearační systém a data, která mají být uchována i po restartu zařízení je třeba uložit na SD kartu, jejíž slot je k dispozici. Protože procesorová jednotka typu ARM11 využíva poměrně zastaralou architekturu ARMv6 s nedostatečnou hardwarovou podporou pro výpočty v polovoucí řádové čárce  \glref{VFP}v2 \cite[ARM11] \fnote{ARM11 Online Technical Reference Manual \url{http://infocenter.arm.com/help/index.jsp?topic=/com.arm.doc.ddi0301h/Cegdejjh.html}}, je nutná rekompilace distribuce systému Debian/Raspbian. Oficiální port této distribuce ARMhf totiž vyžaduje alespoň architektutu ARMv7 s koprocesorem pro výpočty v plovoucí řádové čárce nejméně ve verzi VFPv3-D16. \fnote{Debian, List of official ports \url{https://www.debian.org/ports/}}
 
Druhá verze Raspberry Pi přinesla zvýšení výpočetního výkonu s růstem taktu procesoru na 900Mhz a využitím čtyř výpočetních jader. To vše pod modernější architekturou ARMv7-A s procesorem ARM Cortex-A7 (podpora \glref{VFP}v4 \cite[CA_VFP] \fnote{ARM Cortex-A7 Online Technical Reference Manual \url{http://infocenter.arm.com/help/index.jsp?topic=/com.arm.doc.subset.cortexa.cortexa7/index.html}}) v čipu BCM2836. Tato verze počítače je tak se současnými distribucemi plně kompatibilní. Kromě vyššího výpočetního výkonu došlo i k nárustu hlavní paměti na 1GB. Momentálně je k dispozici jen v modelu B, který navazuje na model B+ verze 1. Na rozdíly jednotlivých variant odkazuje tabulka \ref[RPi_Modely]

Nevýhodou použitých SoC v obou verzích je chybějící integrovaná podpora rozhraní ethernet. Pro připojení do sítě toho typu je použit na desku integrovaný převodník USB-Ethernet.  

%clabel prida referenci - muzu odkazovat \ref + prida do seznamu tabulek / obrazku
%midinsert, topinsert - chce na zacatek stranky, kdyz to nejde, prejde na zacatek dalsi
\midinsert \clabel[RPi_Modely]{Seznam modelů Raspberry Pi}
\ctable{lccccc}{ 
	\hfil Model & A & A+ & B & B+ & Bv2 \crl
		Počet pinů & 26 & 40 & 26 & 40 & 40  \cr
		RAM paměť [MB]  & 256 & 256 & 256/512 & 512 & 1024   \cr
		USB porty & 1 & 1 & 2 & 4  & 4 \cr
		RJ45 & Ne & Ne & Ano & Ano  & Ano \cr
		Slot na kartu & SDHC & MicroSD & SD & MicroSD & MicroSD   \cr
		Příkon [W] & 1.5 & 1.0 & 3.5 & 3 & 4  \cr
		Takt CPU [MHz] & 700 & 700 & 700 & 700 & 900  \cr
		Jádra CPU [MHz] & 1 & 1 & 1 & 1 & 4  \cr
		CPU Arch [W] & ARMv6 & ARMv6 & ARMv6 & ARMv6 & ARMv7-A  \cr
}
\caption/t Seznam modelů Raspberry Pi 
\endinsert

\midinsert \clabel[RPiB+]{Raspberry Pi v1 model B+}
\picw=8cm \cinspic Raspberry_Pi_B+_top.png
\caption/f Raspberry Pi verze 1 model B+
\endinsert
 
 \sec Linux a RT vylepšení
 
 \sec FPGA
 
 Text v sekci.\cite[AGL125] \cite[Libero_ug] \cite[rt-wiki]
 
 \chap Použité řešení
 
 Text v sekci.
 
 \chap Závěr
 

 



 \bibchap
 \usebib/c (simple) mybase

\app Zadání

 \sec pokyny
 
Na platformě procesorové desky Raspberry Pi implementujte systém pro řízení bezkartáčových (BLDC/PMSM) motorů.

1. Pro komunikaci procesorového systému s výkonovým hardwarem realizovaným s využitím programovatelného obvodu (FPGA) vyberte vhodný protokol a periferii.

2. Pro vybraný způsob komunikace navrhněte ovladač na straně jádra Linux a obvodový návrh ve VHDL na straně FPGA.

3. Integrujte bloky pro snímání polohy, řízení výkonových stupňů a měření proudu do FPGA návrhu.

4. S využitím navržených periferií realizujte řízení bezkartáčového motoru.

5. Vyžaduje se podrobná technická dokumentace včetně přípravy podkladů pro prezentaci včetně videozáznamu.

\app Zkratky\par \makeglos %vlozi novou prilohu



 
 
 \bye
  
 