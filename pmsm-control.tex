 \input ctustyle
 \input opmac-bib
 \worktype [B/CZ]
 \faculty {F3}
 \department {Katedra řídicí techniky}
 \title {Řízení bezkartáčových motorů s deskou Raspberry Pi a Linuxem}
 \author {Martin Prudek}
 \date {Duben 2015}
 \abstractEN {Testing}
 \abstractCZ {Testovani}
 \declaration {Prohlasuji, ze jsem pracoval poctive.}
 \makefront
 
 \chap Motivace
 
 Značné zlevnění a zpřístupnění jednodeskových minipočítačů v posledních letech u nás vyvolalo otázku, zda je možné některé hojně rozšířene modely využít k řídicím aplikacím.
 
  Z široké nabídky zařízení jsme vybrali Raspberry PI model B rev. 2.0. U tohoto počítače již byla prostudována možnost řízení motoru stejnosměrného motoru zachytáváním impulzů inkrementálního rotačního senzoru polohy\cite[Meciar]. Takové řízení je však vlivem nedostatečného výpočetního výkonu možné jen do 14kHz (2100ot/min). Proto bylo zpracování hrubých senzorových dat přesunuto do FPGA obvodu, se kterým Raspberry komunikuje prostřednictvím protokolu SPI.
  
  Jednosměrný bezkartáčový motor byl dále nahrazen efektivnějším PMS motorem.
 
 
 \chap PMS motory.

PMS (Permanent Magnet Synchronous) motory jsou díky vysoké efektivitě a robustní kontrukci bez kartáčů vhodnou volbou v mnoha řídicích aplikacích. 

Komutace probíhá na rozdíl od kartáčových motorů elektronicky, což přínáší vyšší požadavky na řídicí hardware. Odměnou je výšší výkon v poměru k váze, stejně tak točivý moment v poměru k příkonu. Výhodou jsou také nížší hlučnost a delší životnost, protože nedochází k opotřebení kartáčů a mechanických částí komutátoru. \cite[PMSM-Kinetis] Elektronická komutace bývá implementována v procesorovém systému, či speciálním obvodu (FPGA / ASIC)\cite[Meloun].
 
 \sec Konstrukce
 
 Text v sekci.
 
 \sec Řízení
 
 
 \chap Popis hardware
 
 Text v sekci.\cite[AGL125] \cite[Libero_ug] \cite[rt-wiki]
 
 \chap Použité řešení
 
 Text v sekci.
 
 \chap Závěr
 
 \sec pokyny
 
Na platformě procesorové desky Raspberry Pi implementujte systém pro řízení bezkartáčových (BLDC/PMSM) motorů.

1. Pro komunikaci procesorového systému s výkonovým hardwarem realizovaným s využitím programovatelného obvodu (FPGA) vyberte vhodný protokol a periferii.

2. Pro vybraný způsob komunikace navrhněte ovladač na straně jádra Linux a obvodový návrh ve VHDL na straně FPGA.

3. Integrujte bloky pro snímání polohy, řízení výkonových stupňů a měření proudu do FPGA návrhu.

4. S využitím navržených periferií realizujte řízení bezkartáčového motoru.

5. Vyžaduje se podrobná technická dokumentace včetně přípravy podkladů pro prezentaci včetně videozáznamu.

 \bibchap
 \usebib/c (simple) mybase
 
 
 
 \bye
  
 