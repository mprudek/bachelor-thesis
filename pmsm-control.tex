 \input ctustyle
 \input glosdata
 \input opmac-bib
 \worktype [B/CZ]
 \faculty {F3}
 \department {Katedra řídicí techniky}
 \title {Řízení bezkartáčových motorů s deskou Raspberry Pi a Linuxem}
 \author {Martin Prudek}
 \date {Duben 2015}
 \abstractEN {Testing}
 \abstractCZ {Testovani}
 \declaration {Prohlasuji, ze jsem pracoval poctive.}
 \makefront
 
 \def\frac#1#2{{\begingroup#1\endgroup\over#2}} %/frac je LaTex, definujme vlastni

 \chap Úvod
 
  Elektrické motory jsou již dlouhou dobu nedílnou součástí moderní civilizace. Jejich řízení se stalo důležitou disciplínou, potřebnou v mnoha aplikacích. Postupem času se přitom stává výhodnějším i pro menší výkony použít složitější elektroniku než řešit komutaci mechanickými kartáči.
  
  S rozvojem technologií a digitálních řidicích zařízení je možné využít automatizaci i tam, kde to dříve nebylo možné. Je to patrné například na stále běžnějším využití ruzných dronů či multikoptér. Zařízení dříve považovaná za přísně střežená vojenská tajemství se dnes, jako už mnohokrát v dějinách, stávají běžnou součástí našich domácností. 
  
  Vývoj se však nepohybuje jen ve směru zdokonalování technologií, neboť v mnoha situacích je rozhodujícím faktorem cena. Návrhář řídicího systému se musí v hojných případech rozhodovat mezi dvěma variantami. Buď použít často velmi levný analogový řídicí obvod, který je ale úzce specializovaný a při úpravě parametrů se jeho modifikace dále velmi prodražují. Nebo řídicí počítač, který při změně problému stačí mnohdy jen přeprogramovat. Na trh se přitom dostává mnoho velmi levných zařízení, z nichž některá jsou myšlena spíše jako hračky či výúkové nástroje. 
  
  Právě taková zařízení, která se již jeví únosně i pro průmyslové nasazení, může být zajímavé propojit s již vyzkoušeným hardware. Tím spíše, pokud na vybraném zařízení pracuje univerzální operační systém - např. GNU/Linux. Tato práce se tak stává studií právě takového řešení.
  
   Z široké nabídky takových zařízení, jako jsou BeagleBone, Raspberry Pi či Banana Pi  
 \fnote{Root.cz, Srovnání: Raspberry Pi a jeho největší konkurenti \url{http://www.root.cz/clanky/srovnani-raspberry-pi-a-jeho-nejvetsi-konkurenti/}} 
 jsme vybrali pro tuto práci Raspberry PI model B rev. 2.0. 
 
 {\bf Raspberry Pi} (dále jen Rpi) je počítač realizovaný na jednom plošném spoji, velikosti kreditní karty, který od roku 2006 vyvíjí britská nadace {\em Raspberry Pi Foundation} \fnote{web Raspberry Pi Foundation\url{https://www.raspberrypi.org/}}. Výhodou je možnost využití univerzálního operačního systému, například výrobcem připraveného sestavení operačního systému GNU/Linux na bázi distribuce Debian. Na takto upravený systém, pojmenovaný Raspbian, je pak možné aplikovat Real-Time modifikaci, která zajistí splnění časových omezení vyžadovaných řízením. Více k Rpi a RT modifikaci Linuxu uvedu v sekci \ref[sec_rpi] a \ref[sec_rt].
 
 U tohoto počítače již byla prostudována možnost řízení stejnosměrného motoru s využitím zpracování impulzů inkrementálního rotačního senzoru polohy \cite[Meciar]. Stejně tak jako u podobných levných zařízení i zde však naražíme na hranice, které nám tento hardware, zvláště s využitím univerzálního OS, nedovoluje překonat. U \glref{RPi} tak docházelo ke kritickým časovým prodlevám při zpracovávání přijatých pulsů, čímž se ztrácela informace o poloze. Takové chování nastávalo již od frekvence pulsů 14kHz, což při 500 pulsech na jednu otočku dělá 2100ot/min. Tedy ve chvíli, kdy perioda příchodu pulsů byla srovnatelná s dobou odezvy systému na tento druh události.
 
 Výdodným řeším tohoto problému se stává přesunutí zpracování pulsů \glref{IRC} do samostatné periferie a zjištěnou polohu motoru posílat jako celek. Rpi tedy zpracovává jen jednu řídicí smyčku o neměnné frekvenci, z hlediska výpočetního výkonu tedy již otáčky motoru nehrají roli. Protože RPi nedisponuje periferíí vhodnou pro tento účel, byla pro tuto práci použita rozšiřující deska vyvinutá firmou {\em PiKRON} \fnote{PiKRON \url{http://www.pikron.com/}}. Tato deska umožňuje návrh potřebné logiky v \glref{FPGA} obvodu. Podrobnosti v sekci \ref[sec_fpga].
 
 Pro propojení rozšiřující desky a nadřazeného systému (RPi) bylo třeba vybrat jednoduchý komunikační protokol. Požadavkem byla možnost obousměrné komunikace a dostatečná rychlost, která umožní vzhledem k frekvenci řídicí smyčky přenést dostatečný objem dat. Bylo také nutné, aby použitý protokol byl natolik nekomplikovaný, že půjde výhodně realizovat v FPGA obvodu a na straně druhé pohodlně zpracovat v jádře operačního systému na RPi. 
 
 Jak jsem již zmínil, překotný vývoj digitálních řidicích zařizení hlavně koncem druhé poloviny 20. století nábídl možnosti jejich využití i tam, kde to dříve nebylo možné. Dalším důležitým milníkem je i pokrok v řízení výkonu a využití nových výkonových spínacích prvků. Příkladem nechť je návr řízení synchronního bezkartáčového motoru s permanentním magnetem v rotoru (dále PMS motory či jen PMSM). Bez digitálního zpracování dat a PWM modulace by jeho provoz nebyl možný. Takový motor přitom nabízí ve srovnáním s konvenčním  DC motorem několik výhod. Odpadá totiž použití kartáčů, nedochází ke kolísání točivého momentu a zvvyšuje se efektivita. Více o PMS motorech uvedu v kapitole \ref[chap_pmsm].

 
 \label[chap_pmsm]
 \chap PMS motory

PMS (Permanent Magnet Synchronous) motory jsou díky vysoké efektivitě a robustní kontrukci bez kartáčů vhodnou volbou v mnoha řídicích aplikacích\cite[ESD]. Zvláště pak v robotice a všude tak, kde je zapotřebí řízení polohy a rychlosti.

Na rozdíl od kartáčových motorů je komutace PMS motorů řízena elektronicky, což přínáší vyšší požadavky na řídicí hardware. Ve chvíli, kdy dnes většina aplikací vyžaduje elektronické řízení jak rychlosti, tak točivého momentu, nepředstavuje ale řídicí elektronika zátěž navíc. \cite[sensorless]. Odměnou je naopak výšší výkon v~poměru k váze, stejně tak točivý moment v~poměru k příkonu. Výhodou jsou také nížší hlučnost a delší životnost, protože nedochází k opotřebení kartáčů a mechanických částí komutátoru. \cite[PMSM-Kinetis] Elektronická komutace bývá implementována v procesorovém systému, či speciálním obvodu (FPGA / ASIC)\cite[Meloun].
 
 3-fázové synchronní motory s permanentním magnetem se často využívají kromě PMS varianty se sinusovým průběhem \glref{BEMF} také v \glref{BLDC} variantě s lichoběžníkovým (trapezoidal) průběhem BEMF. Přičemž výhodou PMSM je konstatní točivý moment v celém rozsahu otáčení, zatímco BLCD motor je snadněji řiditelný a dnes se využívá převážně z historických  důvodů. \cite[sensorless]
 \sec Konstrukce
 
 Základ konstrukce PMS Motoru tvoří rotor s permanentním magnetem a stator, kterým prochází vinutí jednotlivých fází. 
 
 Počet vinutí pak závisí na počtu pólů permanentního magnetu umístěného ve statoru. Obvyklá varianta jsou dva páry pólů (polpáry) permanentního magnetu, viz Obrázek \ref[pole_pairs]. Celý návrh je přitom optimalizován pro buzení sinusovým průběhem, stejně tak zpětné  elktromotorické napětí vykazuje sinusový průběh. \cite[ijetae] \cite[ti_pmsm].
 
\medskip \clabel[pole_pairs]{PMS Motor se dvěma pólovými dvojicemi}
\picw=8cm \cinspic PMSM.png
\caption/f PMS Motor se dvěma pólovými dvojicemi
\medskip 


 \sec Matematický popis
 
 Uvažujme $i_{sa}$, $i_{sb}$ a $i_{sc}$ proudy procházející vinutím statoru, platí:
 
 $$ i_{sa} + i_{sb} + i_{sc} =0\eqmark $$
 
 Toto může být vyjádřeno jako vektor v komplexní rovině, potom:
 
 $$ \overline{i} = i_{sa}+ai_{sb}+a^2i_{sc} \eqmark$$
 kde $a$ a $a^2$ jsou operátory posouvajíci fázi o 120°, $ a=e^{{2j\pi}/{3}} $ a $a^2=e^{{4j\pi}/{3}}$.
 $k$ je transformační konstanta.
 
 $\overline{i}$ můžeme vyjádřit jako součet jeho reálné a imaginární složky:

 $$ \overline{i}=i_{s\alpha}+ji_{s\beta} \eqmark$$
 
\medskip \clabel[proudy]{Komplexní vyjádření vektoru proudu vinutím statoru}
\picw=6cm \cinspic currents.pdf
\caption/f Komplexní vyjádření vektoru proudu vinutím statoru. $\alpha$ značí reálnou osu a $\beta$ značí imaginární osu.
\medskip %obrazek zustal tam, kde ma. \midinsert ho posunul

Se znalostí proudů, protékajících jednotlivými fázemi, pak můžeme vypočítat myšlené proudy tekoucí rovnoběžně s imaginární a reálnou osou.

 $$ 
 \eqalignno{ i_{s\alpha}&=(i_{sa}-i_{sb}/2-i_{sc}/2) \cr
  i_{s\alpha}&=(2i_{sa}-i_{sb}-i_{sc})/3 & \eqmark \cr}
  $$
 
$$
\eqalignno{ i_{s\beta}&=\sqrt{3}(i_{sb}-i_{sc})/2 \cr
	i_{s\beta}&=(i_{sb}-i_{sc})/\sqrt{3}  & \eqmark \cr}
$$

Pro popis PMS motorů je uvažován ideálně symetrický motor se sinusoidně rozloženým vinutím. Pro takovou idealizaci uvažujeme napětí na vinutích $u_{sa}$, $u_{sb}$ a $u_{sc}$  následující:

$$
\label[rce1]
u_{sa}=R_si_{sa}+{{d}\over{dt}}\psi_{sa} \eqmark$$
$$ u_{sb}=R_si_{sb}+{{d}\over{dt}}\psi_{sb} \eqmark$$
$$ u_{sc}=R_si_{sc}+{{d}\over{dt}}\psi_{sc} \eqmark$$

kde $\psi_{sa}$,$\psi_{sb}$ a $\psi_{sc}$ jsou magnetické indukční toky vyvolané proudy odpovídajících vinutí. Vyjádření napětí vektorem v Gaussově(komplexní) rovině (Clarkova transformace, $\alpha\beta$ transformace) bude následující:

 $$ u_{s\alpha}=R_si_{s\alpha}+{{d}\over{dt}}\psi_{s\alpha} \eqmark$$
 $$ u_{s\beta}=R_si_{s\beta}+{{d}\over{dt}}\psi_{s\beta} \eqmark$$
 
 Přitom složky magnetického indukčního toku budou následující:
 
 $$\psi_{s\alpha} = L_{s\alpha}i_{s\alpha} + \psi_Mcos\theta_r \eqmark$$
 $$\psi_{s\beta} = L_{s\beta}i_{s\beta} + \psi_Msin\theta_r \eqmark$$
 
 kde $\theta_r$ je úhlová pozice rotoru a $\psi_M$ je magnetický indukční tok rotoru. $L_{s\alpha}$ a $L_{s\beta}$ jsou složky vzájemné indukčnosti rotor-stator.  

Úhlové zrychlení takového motoru o zátěži $T_L$ s $p$ póly připadajícími na každou fázi můžeme vyjádřit jako:

$$
\label[rce2]
{d\omega\over{dt}}=
{p\over{j}}\lbrack{3\over{2}}p(\psi_{s\alpha}i_{s\beta}-\psi_{s\beta}i_{s\alpha})-T_L\rbrack
\eqmark
$$

Rovnice \ref[rce1] az \ref[rce2] představují model PMS motoru v souřadné soustavě ($\alpha, \beta$) spojené se statorem.


Kromě soustavy spojené se statorem bývá někdy výhodné vyjádřit tyto večiny v soustavě spojené s rotorem.
Pro dosažení maximálních momentových účinků, je totiž výhodné, aby vektor megnetické indukce, magnetického pole vyvolávaného proudy protékající vinutím statoru, svíral pravý úhel s vektorem mg. indukce mg. pole permanentního magnetu rotoru. Pro tyto účely je tedy vhodné veličiny fixovat právě k souřadné soustavě rotoru. 

Pro statickou soustavu jsme úspěchem využili Clarkovu trnsformaci, nyní, pro rotující vztažnou soustavu použijeme Parkovu (dq0) transformaci.

Pro genralizovanou rotující soustavu bude transformace vypadat následovně (viz Obrázek \ref[park]):

$$ \overline{i_{sg}}=\overline{i_s}e^{-j\theta_g} \eqmark$$

\medskip \clabel[park]{Parkova transformace}
\picw=8cm \cinspic StatorCurrents.pdf
\caption/f Parkova transformace 
\medskip 

A pro soustavu spojenou s rotorem již můžeme dosadit:

$$u_{sd}=R_si_{sd}+{d\over{dt}}\psi_{sd}-\omega_r\psi_{sq} \eqmark $$
$$u_{sq}=R_si_{sq}+{d\over{dt}}\psi_{sq}-\omega_r\psi_{sd} \eqmark $$

kde $\psi$ jsou odpovídající mg. indukční toky a $\omega_r$ je úhlová rychlost rotoru.

 $$\psi_{sd} = L_{sd}i_{sd} + \psi_M \eqmark$$
 $$\psi_{sq} = L_{sq}i_{sq} \eqmark$$
 
Úhlové zrychlení: 
 
$$
\label[rce3]
{d\omega\over{dt}}=
{p\over{j}}\lbrack{3\over{2}}p(\psi_{sd}i_{sd}-\psi_{sq}i_{sq})-T_L\rbrack
\eqmark
$$

 \medskip \clabel[dq0]{Poloha os Parkovy transformace}
\picw=8cm \cinspic dq0.pdf
\caption/f Osy d(direct) a q(quadrature) jsou voleny vzhledem k rotoru.
\medskip 

 \sec Řízení
 
 PMS Motory se vyznačují plynulou rychlostí v celém rozsahu rotace a schopností plně řídit točivý moment i při nulové rychlosti. K tomu se využívaji techniky vektorového řízení. Ty rozloží proud protékající vinutím statoru na složky generující magnetické pole a točivý moment. Tyto složky pak můžeme řídit na sobě nezávisle a přiblížit se tak řízení obyčeného kartáčového DC motoru \cite[PMSMC]. Metematickým nástrojem jsou Clarkova a Parkova transformace popsané výše v této kapitole.
 
 
 \chap Popis hardware
 
% Minulá kapitola shrnula teorii týkající se konstrukce bezkartáčových motorů a jejich řízení. Nyní se budu zabývat konkrétným hardware použitým pro tuto práci. Popíši tak parametry použitého motoru, dále Raspberry Pi a jeho rozšiřujíci jednotky.  
 
 \sec Použitý motor
 
 Z široké nabídky bezkartáčových PMS motorů byl vybrán model BLWR233D-36V-4000 od společnosti Aneheim Automation \fnote{Aneheim Automation BLWR23 \url{http://www.anaheimautomation.com/products/brushless/brushless-motor-item.php?sID=148&pt=i&tID=96&cID=22}} viz. Obrázek \ref[motor_obr]. Jedná se o motor s výkonem 92W, pracující při maximálním napětí 36V. Vinutí statoru tvoří 6 polpárů. Vinutí každé z fází tedy tvoří právě dvě pólové dvojice. Analogicky má stator motoru právě dva polpáry, viz. Obrázek \ref[pole_pairs].
 
\medskip \clabel[motor_obr]{Použitý motor}
\picw=8cm \cinspic motor.png
\caption/f Použitý motor 
\medskip  
 
  V motoru jsou integrovány Hallovy sondy, které snímají absolutní polohu motoru ve 12 úsecích mechanického plného kruhu. Při přechodu mezi dvěma úseky se mění práve jeden z výstupních signálů těchto senzorů. V místech přechodu je tedy možné určit pozici motoru relativně přesně.
  
   Do motoru je přivedeno celkem 8 vodičů ve dvou svazcích. Ve svazku o třech vodičích jsou přivedeny fáze, v ostatních je pak napájení a výstup hallových senzorů, viz. Tabulka \ref[vodice_tab]. V Tabulce \ref[motor_tab] jsou pak uvedeny technické parametry motoru.

 \medskip \clabel[motor_tab]{Parametry použitého motoru}
\ctable{lll}{ 
	\hfil Veličina & Uvedené hodnoty / jednotky & Metrický systém  \crl
		Max. napětí & 36 V & 36 V  \cr
		Max. výkon & 92 W & 92 W  \cr
		Max. točivý moment & 31.2 oz-in & 0.0219 Nm  \cr
		Max. otáčky & 4000 rpm & 4000 rpm   \cr
		Torque constant & 8.5 oz-in/A & 0.06 Nm/A  \cr
		BEMF konst. & 4.45 V/krpm  & 4.45 V/krpm   \cr
		Odpor mezi fázemi (line-to-line)& 0.64 $\Omega$ & 0.64 $\Omega$   \cr
		Vzájemná indukčnost fází & 2.1 mH & 2.1 mH    \cr
		Moment setrvačnosti rotoru & 0.00106 oz-in-$sec^2$ & $7.485*10^{-6}$ kg$*m^2$  \cr
		Délka & 2.9 in & 73.7 mm   \cr
		Váha & 1.65 lbs & 0.75 kg  \cr	
}
\caption/t Parametry použitého motoru 
\medskip

 \medskip \clabel[vodice_tab]{Vodiče přivedené do motoru}
\ctable{lll}{ 
	\hfil Svazek & Barva & Funkce  \crl
		1 & Žlutá & Fáze A  \cr
		1 & Červená & Fáze B  \cr
		1 & Černá & Fáze C  \cr
		2 & Červená & Napájení hallů  \cr
		2 & Modrá & Hall senzor A  \cr
		2 & Zelená & Hall senzor B  \cr
		2 & Bílá & Hall senzor C  \cr
		2 & Černá & Uzemění hallů  \cr
}
\caption/t Vodiče přivedené do motoru 
\medskip 
  
  K motoru je též z jedné strany namontován inkrementální rotační čítač (dále jen \glref{IRC}), který měří přírůstky polohy motoru vůči pozici při startu řízení. Jeho výhodou je vysoká přesnost.  Zatímco Hallovy sondy dokáží rozlišit jen mezi 12 polohami na jednu mechanickou otáčku, použitý IRC rozezná 2000 samostatných bodů (běžné hodnoty rozlišení tohoto typu senzorů se pohybují v rozmezí 16 až 10000 bodů na otáčku). K IRC vedou vodiče napájení a signálů s daty o poloze. Dva z nich jsou signály, které ze svých aktuálních stavů a jejich změn poskytují informaci o pohybu motoru a jeho směru. Třetí signál pak vyšle puls vždy jen v jedné pozici za jednu mechanickou otáčku motoru. Poskytuje tak absolutní polohu.
  
 

 
 \label[sec_rpi]
 \sec Raspberry Pi
 
 Raspberry Pi je jednodeskový počítač založený na rodině architektury ARM, který se v současnosti dodavá v několika variantách.
 
 Základem první verze tohoto minipočítače je \glref{SoC} BCM2835, který obsahuje centrální procesor ARM1176JZF-S s taktem 700 MHz, grafický procesor VideoCore IV a 256 MB nebo 512 MB  paměti RAM. Neumožňuje však připojení pevného disku. Operační systém a data, která mají být uchována i po restartu zařízení je třeba uložit na SD kartu, jejíž slot je k dispozici. Protože procesorová jednotka typu ARM11 využíva poměrně zastaralou architekturu ARMv6 s nedostatečnou hardwarovou podporou pro výpočty v polovoucí řádové čárce  \glref{VFP}v2 \cite[ARM11] \fnote{ARM11 Online Technical Reference Manual \url{http://infocenter.arm.com/help/index.jsp?topic=/com.arm.doc.ddi0301h/Cegdejjh.html}}, je nutná rekompilace distribuce systému Debian/Raspbian. Oficiální port této distribuce ARMhf totiž vyžaduje alespoň architektutu ARMv7 s koprocesorem pro výpočty v plovoucí řádové čárce nejméně ve verzi VFPv3-D16. \fnote{Debian, List of official ports \url{https://www.debian.org/ports/}}
 
\medskip \clabel[RPiB+]{Raspberry Pi v1 model B+}
\picw=8cm \cinspic Raspberry_Pi_B+_top.png
\caption/f Raspberry Pi verze 1 model B+
\medskip 
 
Druhá verze Raspberry Pi přinesla zvýšení výpočetního výkonu s růstem taktu procesoru na 900Mhz a využitím čtyř výpočetních jader. To vše pod modernější architekturou ARMv7-A s procesorem ARM Cortex-A7 (podpora \glref{VFP}v4 \cite[CA_VFP] \fnote{ARM Cortex-A7 Online Technical Reference Manual \url{http://infocenter.arm.com/help/index.jsp?topic=/com.arm.doc.subset.cortexa.cortexa7/index.html}}) v čipu BCM2836. Tato verze počítače je tak se současnými distribucemi plně kompatibilní. Kromě vyššího výpočetního výkonu došlo i k nárustu hlavní paměti na 1GB. Momentálně je k dispozici jen v modelu B, který navazuje na model B+ verze 1. Na rozdíly jednotlivých variant odkazuje tabulka \ref[RPi_Modely]

Nevýhodou použitých SoC v obou verzích je chybějící integrovaná podpora rozhraní ethernet. Pro připojení do sítě toho typu je použit na desku integrovaný převodník USB-Ethernet.  

%clabel prida referenci - muzu odkazovat \ref + prida do seznamu tabulek / obrazku
%midinsert, topinsert - chce na zacatek stranky, kdyz to nejde, prejde na zacatek dalsi
\midinsert \clabel[RPi_Modely]{Seznam modelů Raspberry Pi}
\ctable{lccccc}{ 
	\hfil Model & A & A+ & B & B+ & Bv2 \crl
		Počet pinů & 26 & 40 & 26 & 40 & 40  \cr
		RAM paměť [MB]  & 256 & 256 & 256/512 & 512 & 1024   \cr
		USB porty & 1 & 1 & 2 & 4  & 4 \cr
		RJ45 & Ne & Ne & Ano & Ano  & Ano \cr
		Slot na kartu & SDHC & MicroSD & SD & MicroSD & MicroSD   \cr
		Příkon [W] & 1.5 & 1.0 & 3.5 & 3 & 4  \cr
		Takt CPU [MHz] & 700 & 700 & 700 & 700 & 900  \cr
		Jádra CPU [MHz] & 1 & 1 & 1 & 1 & 4  \cr
		CPU Arch [W] & ARMv6 & ARMv6 & ARMv6 & ARMv6 & ARMv7-A  \cr
}
\caption/t Seznam modelů Raspberry Pi 
\endinsert


 \label[sec_rt]
 \sec Jádro Linux a jeho použitelnost pro RT aplikace
 
 Linux je víceuživatelský, víceúhlový operační systém založený na stejnojmenném jádře, které vyvinul v roce 1991 Linus Thorvadls. S vylepšením zaručující maximální čas odezvy na vnější události se stává zajímavou volbou i pro některé řídicí aplikace.
 
 Systém, který byl původně myšlen spíše jako koníček se postupem času vyvinul z pouhého emulátoru terminálu v jeden z nejpoužívanějších na světě. Dnes mají Linuxové distribuce 97\% zastoupení mezi největšími superpočítači \fnote{Linux dominates supercomputers \url{http://www.zdnet.com/article/linux-dominates-supercomputers\|-as-never-before}} a například systém Android, s Linuxovým jádrem, běží na 47\% všech smartphonů a tabletů \fnote{Mobile/Tablet Operating System Market Share \url{http://marketshare.hitslink.com/operating\|-system-market-share.aspx?qprid=8&qptimeframe=M&qpcustomd=1}}. Mohutný rozmach systému byl zapříčiněn mimo jiné zveřejněním jeho zdrojového kódu pod svobodnou licencí. To umožnilo na jeho vývoji pracovat tisícům dobrovolníků a mnoha korporacím, mezi které patří například Red Head, Intel, Samsung či Google \fnote{Who actually develops Linux? \url{http://www.extremetech.com/computing/175919-who-actually-\|develops-linux-the-answer-might-surprise-you}}.
 
 
 Pro široké možnosti svého využití a jednoduché vzdálené správy si systém brzy získal oblibu mezi vyvojáři. Doménou řídicích aplikací však dále zůstavaly menší systémy. Linux se tak stával součástí jen větších celků, kde za odezvu k kritických částech byl zodpovědný \glref{RT} OS a Linux plnil spíše administrativní funkce. Vývoj plikací pro menší RT OS však přinaší mnoho komplikací navíc, přikladem je spíše minimalistické \glref{API} a často nedostatečná podpora některých komunikačních protokolů. Vznikl tak projekt, jehož cílem bylo upravit linuxové jadro způsobem, který umožní včasné přeplánování a zajistí tak splnění časových požadavků vyžadovaných řízením. 
 
 Projekt KURT (Kansas University Real Time) se stal průkopníkem, když využil podporu Linuxu pro víceprocesorové systémy (\glref{SMP}) a rozšířil možnost souběhu procesů při zpracování systémových volání v jádře, což jádro dříve v plné míře nedovolovalo. Serializace takových úseků, ve kterých se v danou chvíli směl nacházet jen jeden procesor je pak řešena zámky s aktivním čekáním (spin-lock). Ostatní úseky kódu s nutností vyloučení vzájemného  souběhu namísto spin-locků začaly využívat RT-mutex. Další snaha byla minimalizovat či přepracovat části kódu, které neumožňují preempci, jako například obslužné rutiny přerušení. Důležitá je implementace dědění priorit.  \cite[preemption] \cite[rychlost_odezvy] 
 
 Vzhledem k rozsáhlosti zdrojového kódu systému Linux není možné analyticky spočítat veškerá zpoždění a vyhodnotit tak maximální latenci systému a časy potřebné k přeplánování. Podobný výpočet by navíc nebyl možný při použití rozmanitých vyrovnávacích pamětí a víceprocesorových systémů. Pro mnoho aplikací je však dostačující znát průběhy zpoždění měřeného na zatíženém systému v delším časovém období - řádově v měsících. Těmito testy a také dalším vývojem RT vlastností Linuxu se zabývá laboratoř OSADL \fnote{OSADL \url{http://www.osadl.org/}}. 
 
 Jedna z posledních verzí RT-varianty jádra (3.18.7-rt2), tak po měsících testování reakcí na vnější události nevykázala byť jediné zpoždění přesahující 100$\mu$s. Přičemž testy probíhaly  na výkonném HW s architekturou x86. Pro některé další architektury, napřiklad ARM pak zpožnějí pro vybrané modely nepřesahuje 200$\mu$s. \cite[rychlost_odezvy].
 
 \secc Aplikace RT patche
 
 \cite[rt-wiki]

 \label[sec_fpga] 
 \sec FPGA obvod 
 
FPGA obvody (Field Programmable Gate Array - Programovatelná hradlová pole) jsou speciální číslicové integrované obvody obsahující různě složité programovatelné bloky, dále násobičky či různé druhy pamětí. Tyto bloky jsou propojené konfigurovatelnou maticí spojů. FPGA obvody se odlišují od naprosté většiny integrovaných obvodů možností přeprogramování - při změně požadavků lze jednoduše náhrát novou konfiguraci.

FPGA se v současnosti využívají v mnoha aplikacích. Těží tak svých vlastností, mezi které patří poměrně snadný návrh v některém z HDL jazyků či grafických nástrojů. Flexibilita, neustále klesající cena ale i zmenšující se spotřeba elektrické energie patří mezi další pozitiva. Jsou tak zajímavou alternativou pro návrh a výrobu integrovaného obvodu na míru, jehož zavedení do výroby bývá velmi nákladné s návratností při výrobě miliónů kusů. Další běžnou oblastí použití je prototypování složitějších zařízení či tvorba periferií pro procesorové jednotky - tzv. \uv{Glue logic}. Složitější FPGA obvody je dokonce možné použít k implementaci procesoru.\cite[fpga_proc]  \cite[xilinx_fpga] \cite[bart_dis]

Pro tuto práci byl vybrán FPGA obvod od společnosti {\em Microsemi Corporation} \fnote{Microsemi Corporation \url{http://www.microsemi.com/}} z rodiny IGLOO. Konkrétně čip AGL125 v pouzdře VQ100. Tato jednotka obsahuje 125 tisíc hradel a nabízí mimo jiné 36 kbits RAM. Podporuje In System Programming (\glref{ISP}) a lze naprogramovat prostřednictvím rozhraní \glref{JTAG} \cite[AGL125]. 

Na rozdíl od větších a výkonnějších obvodů Xilinx a Altera zde dochází k uložení konfigurace přímo do obvodu a neni ji tak třeba nahrávat z externi FLASH či z CPU. Taktéž poměr ceny a kvality produktu je příznivý. Pro inicializaci hodnot v obvodu při zapnutí napájení pak může být využit externí signál. Výhodným řešením může být i využití resetu z bloku fázového závěsu. Bez implementace tohoto mechanismu je ale třeba brát na zřetel hrozbu inicicializace na nevhodné hodnoty. Přikládem jsou signály pro ovládání polovičních H-můstků, které, při inicializaci na logickou jedničku mohou zapříčinit přehřátí a zničení motoru.  

Návrh byl vytvořen v HDL jazyce VHDL. Jazyk VHLD slouží jak k popisu a následné syntéze obvodu v programovatelném hradlovém poli, tak k simulacím. Používá se také pro popis obvodů, které se podle návrhu ve VHDL později přímo vyrobí.  Standardizován byl v r. 1987. Jako jazyk vyvinutý z přísně typového jazyka Ada si pak ponechává některé jeho vlastnosti, které se s výhodou využijí při popisu hardware. Výhodou jazyka Ada, tedy i VHDL, je zabudovaná možnost souběžného výkonávání kódu. Tato funkce se totiž plně využije právě při navrhování hardware, kde je souběžné vykonávání elementární záležitostí. Vznikají tak bloky {\em Process}, které po příchodu spouštěcího signálu pracují zcela paralelně. 

\sec Syntéza VHDL kódu a programování FPGA

Pro syntézu kódu jsem použil balíček aplikací {\em Libero} viz. Manuál \cite[Libero_ug] vyvíjený výrobcem obvodu. Program je možné používat i zdrama. Jedinou podmínkou je registrace a získání \uv{FREE Licence}. Před každým spuštěním aplikace pak musí běžet licenční daemon. Jeho spouštěcí příkaz je následující:
\begtt
/opt/microsemi/licensing_daemon/lmgrd ...
-c /opt/microsemi/license/license.dat ...
-l /home/user/license.log
\endtt

Samotná syntéza se pak provede spuštěním skriptu "synthetize-agl.sh". Tento skript volá program "synplify_pro", který je součástí balíčku staženého s aplikací Libero. Všechny soubory VHDL, včetně nejvyšší entity, musí být uvedené v souboru "syn.tcl" a při syntéze musí být umístěné v spolu se skriptem v jednom adresáři.

Pro naprogramování FPGA obvodu je využívána open-source aplikace  {\em UrJTAG} hostovaná na serveru {\em Sourceforge} \fnote{UrJTAG, Sourceforge.net \url{http://sourceforge.net/projects/urjtag/}}. Pro běh aplikace je nutné mít nainstalovány knihovny "libusb" a "libftdi". Dále je  možné FPGA obvod naprogramovat spuštěním skriptu "program-agl.sh" ze stejného adresáře, jako proběhla syntéza.

\label[vyk_hw_sec]
\sec Rozšiřující jednotka

Rozšiřující jednotka k RPi je tvořena dvojicí plošných spojů vyvinutých firmou {\em PiKRON}. Jsou zde integrovány bloky pro výkonové řízení motoru a měření proudů protékajících jeho vinutím. V tomto místě také dochází ke zpracování výstupů z IRC a Hallových senzorů. Pro ochranu řidíci elektroniky a potlačení rušení je výkonová část kompletně galvanicky oddělena od napájení RPi a FPGA obvodu. Schéma se nachází v příloze \ref[priloha_motor].
  
Výkonové řízení motoru realizuje trojice polovičních H-můstků řady LT1158 \fnote{Half Bridge N-Channel Power MOSFET Driver \url{http://cds.linear.com/docs/en/datasheet/1158fb.pdf}}, které jsou připojeny jako budiče k výkonovým N-MOS tranzistorům. Galvanické oddělení zajišťují dvoukanálové číslicové izolátory \fnote{ADuM1200 Dual-Channel Digital Isolators \url{http://www.analog.com/media/en/technical-\|documentation/data-sheets/ADuM1200_1201.pdf}}. 

Pro měření proudu přímo jednotlivými fázemi motoru je použita sada senzorů využívajících Hallova efektu. K odečítání napěťové úrovně senzorů je určen 12-bitový AD převodník \fnote{ ADS7841 \url{http://www.ti.com/lit/ds/symlink/ads7841.pdf}}. Tento integrovaný obvod je vybaven čtyřmi kanály, z nichž zde jsou použity tři. Pro přepínání mezi nimi je v převodníku integrován multiplexor. ADC může pracovat při napájecím napětí v rozsahu od 2.7V do 5V. Od napájecího napětí, které je současně i referenčním (viz. Příloha \ref[priloha_motor]), se následně odvíjí frekvence hodinového signálu, který je do obvodu přenášen. Při $U_{ref}$=2.7V je maximální frekvence 2Mhz, pro $U_{ref}$=5 pak $f_{CLK}$=5Mhz \cite[adc_man].

 Jedná se o převodník s postupnou aproximací. Tento druh převodníku vyžaduje pro každý převod pevně stanovený počet hodinových cyklů. Při každém hodinovém cyklu pak teoreticky dochází k zpřesnění o jeden bit. Realizován bývá obvykle s využitím několika komponent. Jednou z nich je jednoduchý analogový vzorkovač, který udrží měřenou veličinu neměnnou po čas nutný pro konverzi. Dalšími součástmi jsou DA převodník a aproximační registr(SAR). Komunikací s AD převodníkem se podrobně věnuji v sekci \ref[adc_sec].

\medskip \clabel[propojeni_obr]{Propojení Raspberry Pi s výkonovým HW}
\picw=10cm \cinspic propojeni.png
\caption/f Propojení Raspberry Pi s výkonovým HW
\medskip 

 Celá jednotka může být napájena v poměrně velkém rozsahu napětí pohybujím se mezi 9V až 30V. Toto napětí je poté konvertováno pomocí DC-DC měniče \fnote{TEN 6-2411WIN  Traco Power DC/DC Converters \url{http://mediaserver.voxtechnologies.com/FileCache/Traco\%20Power-TEN\%206WIN\%20Series-datasheet1-1243850956.pdf}} na hodnotu 5V, které je vyžadováno RPi. Protože logika na RPi využíva 3.3V napěťovou úroveň, dojde zde ještě ke konverzi na 3.3V, kterými je napájen i FPGA obvod. Mechanické propojení rozšiřující jednotky s Raspberry Pi je realizováno prostřednictvím plochého 26-žilového kabelu viz. Obrázek \ref[propojeni_obr].



 
 
 
 \chap Použité řešení
 
 V předcházející kapitole jsem se věnoval použitému hardware a jeho specifikacím. Nyní popíši, jak probíhala práce samotná. 
 
 První sekce se zabývá výběrem a implementací vhodného komunikačního protokolu, který slouží k propojení rozšiřující jednotky s nadřazeným hardware. Další část se věnuje čtení a částečnému zpracování dat z A-D převodníku. Ten je připojen k senzorům snímajícím proud jednotlivými fázemi motoru. Budu hovořit také o požadavcích RT rozšíření Linuxu a operacím, které je nutné pro jeho správnou činnost provést. Tématem bude i generování PWM signálů a dekódování IRC. Nakonec popíši elektronickou komutaci.
 
 \sec Frekvence řídicí smyčky
 
 Ve chvíli, kdy přejdeme od spojitého řízení k diskrétnímu, je nutné se zabývat frekvencí řídicí smyčky. Řídici smyčka je proces, při kterém dochází k periodickému vzorkování hodnot nutných pro řízení a následnému nastavení velikosti akčního zásahu. Její frekvence je tak závislá na časových konstantách řízeného systému, či zjednodušeně, jen na té dominantní. V případě elektrických motorů je přitom dominantní časovou konstantou ta mechanická.
 
 U malých elektrických motorů se mechanická časvá konstanta pohybuje v rozmezí od 1 do 20ms. Je ale obvyklé, že i moderní motory s lehkou kotvou nemají menší hodnoty mechanické časové konstanty než 3ms. \cite[rychlost_odezvy]. Pro přesné určení časových konstant je třeba brát v úvahu parametry motory uvedené v manuálu nebo v tabulce \ref[motor_tab].
 
 Motor, kde je vyžadovánoi poziční řízení se chová jako integrátor. Pokud chceme od regulátoru kvalitní výsledky, především krátkou dobu ustálení, je tak třeba volit frekvenci regulační smyčky alespoň dvojnásobnou v porovnání s vlastní frekvencí motoru. V mezním případě můžeme vlastní frekvence dosahovat hodnot $f={{1}\over{0.003}}=333$Hz. S dostatečnou rezervou tak byla zvolena frekvence regulační smyčky o velikosti 1kHz.
 
 \sec Objem dat přenášených mezi RPi a rozšiřující jednotkou
 
 K úspěšnému řízení potřebuje řídící jednotka (RPi) data získaná ze senzorů proudu, IRC a Hallových sond. Rozšiřující jednotka pak vyžaduje hlavně šířku plnění PWM a řídicí signály pro poloviční H-můstky. Je také nutné určit přesnost, s jakou se budou potřebná data přenášet. 
 
 Z IRC je posíláno 32 bitů dat s informací o aktuální poloze motoru. Dále je posílána 12-bitová informace o vzdálenosti od indexu. Index je přesně určené místo v rozsahu otáčení, maximální hodnota této vzdálenosti tak může být 1999 bodů. Tři byty jsou rezervovány pro výstupy z Hallových sond. Následuje 3x16 bitů informace o změřených proudech. Celkem tak v tomto směru dojde při každém průchodu řídicí smyčkou k výměně 128 bitů dat.
 
 Ke výměně stejného objemu dat dojde z důvodu použitého komunikačního protokolu i v opačném směru. Z RPi do rozšiřující jednotky přichází tři 11-bitové informace o šířce plnění PWM. Tyto hodnoty jsou ale z důvodu snadnějšího použití zarovnány na 16 bitů. Jen dolních 11 z nich je ale využito. Součástí přenosu jsou ještě 3 bity odemykající poloviční H-můstky a 3 bity určené k jejich úplné deaktivaci. Poslední využitý bit pak při změně z logické 0 do 1 povoluje resetování AD převodníku. Celkem je v tomto směru komunikace využito jen 40 ze 128 bitů.
 
 
 \sec Komunikační protokol
 
 Výběr komunikačního protoku mezi RPi a FPGA obvodem se stal jednou z prvních otázek a bylo ji třeba zodpovědět již na začátku práce. 
 
 Existuje celá řada známých a hojně využívaných protokolů pro komunikaci ať již mezi periferiemi a procesorovou jednotkou či mezi procesory navzájem. Příkladem je $I^2C$, SSI, USART a jeho synchronní varianta SPI. Požadavkem přitom bylo, aby komunikační protokol byl dostatečně rychlý a aby při frekvenci minimálně 1kHz dokázal přenést alespoň 128bitů. A to oběma směry. Je také nutné, aby byl dostatečně jednoduchý pro efektivní realizaci v FPGA obvodu.
 
 Protokol $I^2C$ se zdál příliš složitý, vzhledem ke svým módům. Adresace, kterou poskytuje by mohla být využita v případě ovládání více os, zde je zbytečná. Hlavní nevýhodou je nedeterministické chování. Asynchronní verze protokolu USART nemá v tomto případě smysl, protože náš FPGA obvod nemá vnitří zdroj hodinového signálu. Hlavní doménou tohoto protokolu je využití pro konzoli ajeho složitá implementace opět nezaručuje determinismus. Jako příhodné varianty se jeví protokoly SSI a SPI. Výhodou SSI protokolu je, že jím generovaný hodinový signál, posílaný po samostatném vodiči, není zastaven s koncem komunikačního cyklu. Protokol SPI hodinový signál v této chvíli zastavuje a je tedy nutné do FPGA obvodu přivést hodinový signál jinou cestou. Tento signál je totiž nutný pro funkci všech synchronních komponent obvodu. Na rozdíl od protokolu SSI je ale SPI integrován (alespoň jeho master varianta) do jádra operačního systému Linux. Není tedy třeba tvořit vlastní ovladač. Výhodou SSI ale zůstává časté použití při připojování obvodů k DSP a pro motion control. Pro práci byl nakonec vybrán protokol SPI.
 
\midinsert \clabel[spi_obr]{Propojení Raspberry Pi s výkonovým HW}
\picw=10cm \cinspic SPI_single_slave.pdf
\caption/f SPI s jedením uzlem typu slave (Zdroj:wikipedia.org)
\endinsert 

SPI je deterministický protokol využívající model komunikace \uv{master-slave}. Pro komunikaci mezi dvěma zařízeními jsou vyžadovány obvykle čtyři datové vodiče. "SCLK" přenaší hodinový signál, jehož zdrojem je master. "MOSI" (Master Out Slave In) je datová linka nesoucí sekvenci bitů směrem od mastera ke slave. "MISO" (Master In Slave Out) pak vede data opačný směrem. Čtvrtým vodičem je "SE" (Slave Enable, někdy značeno "CE" - Chip Enable, na slave zařízeních "SS" - Slave Select), obvykle řízený inverzní logikou (Active-Low), kterým master oznamuje slave, že spustí hodinový signál a dojde k výměně dat. V případě více uzlů slave vede do každého zvláštní vodič "SE", zatímco ostatní tři jsou společné. 

Existují čtyři módy tohoto protokolu, které jsou jsou kombinací dvou bitů \cite[spidev_sum]. První bit, "CPOL" oznamuje, jaký je stav hodin (signálu "SCLK") na počátku komunikace - ve chvíli, kdy "SE" přejde do stavu "Low". Při "CPOL"=0 je počáteční stav "SCLK"=0 a prví hrana je tedy nástupná. Při "CPOL"=1 začínají hodiny ve stavu "High" a první hrana je sestupná. Druhý bit, "CPHA" indikuje, zda jsou data vzorkována či zapisována při náastupné nebo sestupné hraně. Při "CPHA"=0 jsou data vzorkována při nástupné hraně "SCLK" a zapisována při hraně sestupné. Pokud je "CPHA"=1, je tomu naopak. Platí přitom, že zápis dat, stejně jako jejich vzorkování, probíhá v obou zařízeních současně. Číslování módů shrnuje Tabulka \ref[spi_tab].


 \medskip \clabel[spi_tab]{Módy protokolu SPI}
\ctable{ccc}{ 
	\hfil Mód & CPOL & CPHA  \crl
		0 & 0 & 0  \cr
		1 & 0 & 1  \cr
		2 & 1 & 0  \cr
		3 & 1 & 1  \cr
}
\caption/t Módy protokolu SPI 
\medskip 

Při použití módů 0 a 2, kdy "CPHA"=0 je nutné data připravit ještě před příchodem první hrany hodin. Ta, ať už je sestupná či nástupná, totiž data již vzorkuje. Data jsou tedy vystaveny s poklesem signálu "SE". Průběhy komunikace při jednotlivých kombinacích "CPOL" a "CPHA" ukazuje Obrázek \ref[spi_mods_obr]. 

\midinsert \clabel[spi_mods_obr]{Průběh komunikace po SPI při různých módech}
\picw=9cm \cinspic SPI_timing_diagram2.pdf
\caption/f Průběh komunikace po SPI při různých módech (Zdroj:wikipedia.org)
\endinsert 

Protože jednotlivé módy jsou kvalitativně rovnocené, výběr jednoho z nich není nijak rozhodující. Přesto se dá říci, že mód 0 je asi nejpoužívanější, proto padla volba právě na něj. Dále v textu nebudu tuto problematiku více zmiňovat a budu předpokládat, že používáme právě mód 0. Šířka přenášeného slova pak byla zvolena na osm bitů.  

\secc Zdroj hodinového signálu pro FPGA

Jak již bylo řečeno, hodinový signál generovaný při použití protokolu SPI trvá jen po dobu přenosu dat. Zdrojem hlavního hodinového signálu pro FPGA obvod se tak stal generátor hodinových pulzů na RPi, jehož výstup me možné přivést na jeden ze vstupně/výstupních pinů. Jeho frekvence byla stanovena na 50MHz. Dále v textu bude tento signál označen stejně, jako ve VHDL kódu, tedy "gpio_clk". 

\label[spi_slave_sec]
 \secc SPI slave v FPGA obvodu
 
 FPGA obvod musí být v tomto řešení uzlem typu slave. Uzel typu master je zdrojem hodinového signálu, který, jak bylo uvedeno dříve, náš obvod sám vygenerovat neumí. Implementace je součástí souboru "rpi_pmsm_control.vhdl". 
 
Komunikace je pak součástí jednoho Procesu v jazyce VHDL. Proces byl vytvořen jako synchronní, spínaný jen nástupnou hranou vnitřního hodinového signálu ("gpio_clk"). Spínání tohoto procesu signály "SCLK" a "SE" by nebylo výhodné, protože by vedlo k vytvoření nové hodinové domény. Řešení by tak bylo náchylné na hazardy s tím spojené. Je tedy nutné signály "SE" a "SCLK" synchronně vzorkovat. V kódu Procesu se tak objeví následující část:

\begtt
	wait until (gpio_clk'event and gpio_clk='1');		
		  --SCLK edge detection
		  spiclk_old(0)<=spi_clk;
	 	  spiclk_old(1)<=spiclk_old(0);
			
		  if (spiclk_old="01") then --rising edge, faze cteni
		  ...				
		  práce s daty		
		  ...
		  end if;
\endtt
 
 Protokol SPI je v FPGA obvodu obvykle implementován jako posuvný registr viz. Obrázek \ref[spi_trans_obr]. S aktivací signálu "SE", čili s jeho sestupnou hranou, jsou připraveny data do posuvného registru. Zároveň je třeba do vodiče "MISO" zapsat první bit přenosu. První příchozí hrana signálu "SCLK" je nástupná, tedy vzorkovací. Po jejím příchodu je posuvný registr posunut o jeden bit doleva a na uprázdněné místo je zapsán aktuální stav vodiče "MOSI".
 
 \begtt
 dat_reg(127 downto 0) <= dat_reg(126 downto 0) & spi_mosi;
 \endtt
 
 Po příchodu druhé hrany signálu "SCLK", tedy sestupné, je nutné vystavit do vodiče "MISO" nový bit.
 
  \begtt
 spi_miso <= dat_reg(127); 
 \endtt
 
\midinsert \clabel[spi_trans_obr]{Realizace SPI pomocí posuvného registru}
\picw=9cm \cinspic SPI_8-bit_circular_transfer.pdf
\caption/f Realizace SPI pomocí posuvného registru (Zdroj:wikipedia.org)
\endinsert 

Tento postup se opakuje až do deaktivace signálu "SE".
 
 \secc SPI master na Raspberry Pi
 
 Pro uživatelské aplikace je periferie SPI v OS Linux zpřístupněna jako znakové zařízení (obvyklé řešení pro Unix nebo Windows NT). Po otevření určeného souboru/inode reprezentujícío SPI je možné přenosy realizovat voláním řídicí funkce(systémového volání IOCTL) s určenými parametry. Implementace znakového zařízení je umístěna v driveru "spidev.c", který je součástí jádra OS Linux.
 
 V testovacím kódu je pak pužití tohoto driveru součástí souboru "rp_spi.c". V úvodu je třeba vložit hlavičku "#include <linux/spi/spidev.h>". Klíčovým prvkem je zde volání zminěné funkce "int ioctl(int d, int request, ...);". První argument, argument "d" je file descriptor znakového zařízení. Ten je možné získat pomocí funkce "int open(const char *pathname, int flags);" , přičemž argumentem "pathname" je cesta k souboru. Argument "flags" určuje mód otevření. Příkaz tedy může vypadat:
 \ttline+0
 \begtt
 int fd = open("/dev/spidev0.1", O_RDWR);\endtt
 
 Makro "O_RDWR" specifikuje otev5ení pro čtení i psaní.
 
 Druhým argumentem funkce "int ioctl(int d, int request, ...);" je request kód. Jedná se o kód specifický pro každou operaci s daným zařízením zvlášť. Posledním argumentem je konečně pointer na úsek alokované paměti. Není nutné volit pointer nakonkrétní datový typ. Konvence však, tam, kde není uvedeno jinak, doporučuje volit "char * argp" \cite[ioctl_man].
 
 Jako první je třeba nastavit mód SPI, to je provedeno s makrem "SPI_IOC_WR_MODE" následovně:
  \begtt
  unsigned char mode = 0;
  int ret = ioctl(fd, SPI_IOC_WR_MODE, &mode);\endtt
  
 Dále délku slova, využijeme "SPI_IOC_WR_MODE":
 \begtt
 unsigned char bits = 8;
 ret = ioctl(fd, SPI_IOC_WR_BITS_PER_WORD, &bits);\endtt
  
  Poslední část konfigurace vyžaduje nastavit frekvenci hodin "SCLK", použijeme makro "SPI_IOC_WR_MAX_SPEED_HZ". Frekvence je očekávána v jednotkách Hz.:
  
   \begtt
 unsigned int speed = 500000;
 ret = ioctl(fd, SPI_IOC_WR_MAX_SPEED_HZ, &speed);\endtt
  
  Nyní je zařizení SPI plně nakonfigurováno a může začít přenos dat. Ten je proveden z využitím struktury "spi_ioc_transfer":
  \ttline+9 
  \begtt
  struct spi_ioc_transfer tr = 
  {
		.tx_buf = (unsigned long)tx,
		.rx_buf = (unsigned long)rx,
		.len = 16,
		.delay_usecs = 0,
		.speed_hz = speed,
		.bits_per_word = bits,
  };\endtt
  , kde "tx" je pointer na odesílaná a "rx" na příjmaná data, jejichž délka musí odpovídat velikosti vyměněných dat: 
 \ttline+7  
  \begtt
  unsigned char rx[16];
  unsigned char tx[16];\endtt

\secc Pořadí dat v přenosu
 
  Prvním přenášeným bitem komunikace je bit s nejvyšší hodnotou (MSB), tedy bit "dat_reg(127)" na straně FPGA. Za předpokladu, že data pro přenos jsou v Raspberry uchovávána v poli charů a odpadnou tak problémy s endianitou (instrukce ARMv6 jsou defaultně little-endian \fnote{ARMv6 Support for mixed-endian data \url{http://infocenter.arm.com/help/index.jsp?topic=/com.arm.doc.ddi0301h/Cdfbbchb.html}}), bude prvním přenášeným bitem MS bit v "tx[0]". Funkce bitů v přenosech z Raspberry do FPGA a v opačném směru ukazují Tabulka \ref[data_order1_tab] a Tabulka \ref[data_order2_tab].
  
 \medskip \clabel[data_order1_tab]{Pořadí dat v přenosu}
\ctable{ccc}{ 
	\hfil Bity & Byte přenosu & Funkce \crl
		127 & tx[0] &Aktivace resetu AD převodníku \cr
		126 .. 124 & tx[0] & Enable PWM1 .. PWM3 \cr
		123 .. 121 & tx[0] & Shutdown můstku pro PWM1 .. PWM3 \cr
		120 .. 43 &  &- \cr		
		42 .. 32 & tx[10] tx[11] & Šířka plnění PWM1 \cr
		31 .. 27 & &- \cr
		26 .. 16 & tx[12] tx[13] & Šířka plnění PWM1 \cr
		15 .. 11 & &- \cr		
		10 .. 0 & tx[14] tx[15] & Šířka plnění PWM1 \cr
}
\caption/t Přenos z Raspberry Pi do FPGA obvodu 
\medskip 

 \medskip \clabel[data_order2_tab]{Pořadí dat v přenosu}
\ctable{ccc}{ 
	\hfil Bity & Byte přenosu & Funkce \crl
		127 ..96 & rx[0] .. rx[3] & IRC \cr
		95 & rx[4] & Hall 1 \cr
		94 & rx[4] & Hall 2 \cr
		93 & rx[4] & Hall 3 \cr
		92 .. 81 & rx[4] rx[5] & Pozice indexu\cr
		80 .. 72 & rx[5] rx[6] & Počet sečtených proudů \cr
		71 .. 48 & rx[7] .. rx[9] & Součet proudů, kanál 2 \cr
		47 .. 24 & rx[10] .. rx[12] & Součet proudů, kanál 0 \cr
		23 .. 0 & rx[13] .. rx[15] & Součet proudů, kanál 1 \cr
}
\caption/t Přenos z FPGA obvodu do Raspberry Pi 
\medskip 
  
  \label[adc_sec]
 \sec Čtení hodnot z AD převodníku 
 
Znalost proudů tekoucích jednotlivými fázemi motoru je nutná k tzv. Sensorless řízení a je součástí i dalších výpočtů. 

V plošném spoji výkonového stupně {\em 3P-MOTOR-DRIVER-1} je proto integrován čtyřkanálový, 12-bitový AD převodník ADS7841. Na vstup převodníku jsou připojeny senzory proudu protékajícího fázemi, které pracují na principu Hallova efektu. Převodník může pracovat buď ve 12-bitovém nebo jen v 8-bitovém režimu \cite[adc_man]. 

Důležité je zvolit správnou hodinovou frekvenci, která je závislá na napájecím napětí viz sekce \ref[vyk_hw_sec]. Obvod je stejně, jako celé FPGA, napájen napětím 3.3V, které je přivedeno z RPi. Hodinová frekvence tak byla stanovena na hodnotu 2.08Mhz. Tato frekvence je vyvořena kombinací frekvenčních děliček implementovaných v komponentách "divider.vhdl" a "adc_reader.vhdl". V komponentě "dvider.vhdl" dochází k vydělení frekvence 50MHz dvanácti a vzniká tak frekvence o velikosti 4.17MHz. Tato frekvence je následně vydělena dvěma v komponentě "adc_reader.vhdl", o které je více napsáno níže. Aby nedošlo k vytvoření nové hodinové domény, je ke spouštění Procesu ve VHDL určen hlavní 50MHz hodinový signál. Tento signál pak pouze vzorkuje pomalejší 4.17Mhz hodiny. Podobné řešení je popsáno již v sekci \ref[spi_slave_sec].

AD převodník je schopen pracovat v několika módech, které se liší počtem hodinových cyklů nutných k jednomu převodu. V základním módu je třeba 24 hodinových pulzů na konverzi. Tento mód předpokládá komunikace se zařízením vyžadujícím komunikaci po SPI protokolu s pevnou délku slova 8 bitů. Je určený především pro nepravidelné odečítání většinou z mikrokontroléru. Pokud je třeba odečítat hodnoty pravidelně a co nejrychleji, je k dispozici 16-pulzový převod.
Máme-li ale možnost použít FPGA obvod, nemusíme se omezovat 8-bitovou šířkou slova a můžeme využít nejrychlejší variantu. Tedy konverzi dokončenou za 15 hodinových pulzů. V našem případě byla zvolena právě tahle varinta. Průběh takového převodu ukazuje Obrázek \ref[15clkconv_obr].

\medskip \clabel[15clkconv_obr]{Průbeh konverze v režimu 15 hodinových pulzůp na převod}
\picw=13cm \cinspic 15clkconv.png
\caption/f Průbeh konverze v režimu 15 hodinových pulzůp na převod (Zdroj:Texas Instruments)
\medskip

Komponenta "adc_reader", kromě frekvenční děličky, která je spíše vedlejším produktem, realizuje celou komunikaci mezi AD převodníkem a FPGA. Ta se skládá z přenosu řídícího slova z FPGA do ADC a z následného vyčtení hodnot. Převodník je připojen pomocí čtyř vodičů, analogických k vodičům použitých v SPI komunikaci: "CS"(SE), "DCLK"(SCLK), "DIN"("MOSI") a "DOUT"("MISO").

Řídící slovo se sestává z osmi bitů. Prvním z nich je startbit, který při delším provozu s nepravidelným vyčítáním, při kterém není  vypnutý hodinový signál, definuje počátek komunikace. Následují tři bity adresy určené pro výběr jednoho ze čtyř kanálů. Další bity definují výběr buď 8 či 12-bitové konverze a volbu mezi  \uv{Single-Ended} nebo Diferenčním zapojením. Poslední dva bity slouží k nastavení úspornéro režimu v době mezi konverzemi.  

Komponenta "adc_reader" pak implementuje komunikaci pomocí stavového automatu. Každý ze stavů generuje buď logickou úroveň 1 či 0 do signálu "DCLK". Pokud tak nastala při přechodu mezi dvěma stavy sestupná hrana,   je zapsán bit řídícího slova na vodič "DIN". Ve stejné chvíli také dochází k zápisu hodnoty na vodič "DOUT" na straně ADC. Při nástupné hraně jsou naopak hodnoty na vodičích "DIN" a "DOUT" vzorkovány. Protože k přechodu mezi dvěma stavy dochází vždy synchronně s nástupnou hranou hodinového signálu, má signál "DCLK" vysílaný do převodníku práve poloviční frekvenci.

Nutnost přecházet mezi stavy však neexistuje jen u konfigurace převodníku a vyčítání hodnot. Je třeba totiž přepínat i mezi jednotlivými kanály. Ve výše zmíněném stavovém automatu je tak vnořen ještě jeden, který přepíná právě mezi odečítáním proudů z jednotlivých fází.

Konfiguraci řídicího slova a adresy kanálů všech tří fází ukazují tabulky

 \medskip \clabel[ctrl_word_tab]{Konfigurace řídícího slova}
\ctable{lcl}{ 
	\hfil Bit & Hodnota &Funkce  \crl
		7 (MSB) & 1 & Start bit \cr
		6 & Bit 2 adresy & Bit 2 adresy \cr
		5 & Bit 1 adresy & Bit 1 adresy \cr
		4 & Bit 0 adresy & Bit 0 adresy \cr		
		3 & 0 & 12-bitový převod \cr
		2 & 1 & Single-Ended zapojení \cr
		1 & 0 & Power Down mód \cr
		0 (LSB) & 0 & Power Down mód \cr
}
\caption/t Přenos z FPGA obvodu do Raspberry Pi 
\medskip 

 \medskip \clabel[address_tab]{Konfigurace řídícího slova}
\ctable{cccc}{ 
	\hfil Kanál & Bit 2 adresy & Bit 1 adresy & Bit 0 adresy  \crl
		ch0 & 0 & 0 & 1 \cr
		ch1 & 1 & 0 & 1 \cr
		ch2 & 0 & 1 & 0 \cr
}
\caption/t Přenos z FPGA obvodu do Raspberry Pi 
\medskip 

\secc Sčítání více hodnot proudu

Po první implementaci měření proudů bylo patrné, že nedochází k ustálení hodnot přibližně na posledních 3 bitech. Částečným zpřesněním může být použití výběrového průměru namísto jen jedné změřené hodnoty.

 K vyčtení hodnoty z AD převodníku dochází vzhledem k ferekvenci řídicí smyčky poměrně rychle. Při 15 hodinových cyklech nutných pro jednu konverzi a hodinové frekvenci 2.08Mhz, může dojít až ke 138 převodům. Na každou měřenou fázi to dělá zhruba 46 odečtených hodnot. Byla provedena následující úvaha:

Změřenou hodnotu proudu procházejícího jednou z fází za neměnných podmínek označme jako náhodnou veličinu $X$. Za předpokladu, že tato náhodná veličina má normální rozložení pravděpodobnosti, platí, že skutečná hodnota proudu je rovna její střední hodnotě. Tedy $i_s=EX$.

Realizujme tedy náhodný výběr z této veličiny a spočtěme jeho výběrový průměr. Pro výběrový průměr náhodné veličiny platí následující vztahy\cite[navara]:

Výběrový průměr z náhodného výběru ${\bf{X}}=(X_1,...,X_n)$ je

$$\overline{\bf{X}}={{1}\over{n}}\sum\limits_{j=1}^n X_j \eqmark$$

Potom platí, že střední hodnota výběrového průměru se rovná střední hodnotě náhodné veličiny $X$.

$$ E\overline{\bf{X}}=EX \eqmark $$

Pro jeho směrodatnou odchylku $\sigma_{\overline{\bf{x}}_n}$ pak platí:

$$\sigma_{\overline{\bf{x}}_n}={{1}\over{\sqrt{n}}}\sigma_x \eqmark$$

Za předpokladu, že $n=46$ tedy platí, že $\sigma_{\overline{\bf{x}}_{46}}\approx{{1}\over{6.8}}\sigma_x$. Směrodatná odchylka výběrového průměru se oproti odečtení samostatných měření zmenšila 6.8x. A protože $\log_n 6.8\approx2.7$, můžeme říci, že jsem získali téměř tři bity informace navíc.

\sec Generování PWM

O generování signálů PWM se stará komponenta "mcpwm". Návrh implementuje 12-bitovou PWM. Základní myšlenkou je srovnání signálních vektorů "count" a "match". "count" je výstup z 12-bitového čítače, který je spouštěný hlavním 50MHz hodinovým signálem. "match" je pak odesílnáno ve finální podobě již z Raspberry Pi. Frekvence PWM je tedy $50000/2^{11}\approx 24.4$kHz.

Na začátku cyklu, který je detekován při přetečením counteru signálem "sync", je tedy výstup PWM nastaven na 1. Zároveň je kontrolována podmínka "match=count", při jejím splnění dojde nastavení výstupu PWM na 0. Jeden hodinový cyklus před nastavením signálu "sync" na hodnotu 1, je prováděna kontrola, zda je k dispozici nová hodnota signálu "match". V takovém případě je v dalším cyklu PWM používána již aktualizovaná hodnota. 

Protože potřebujeme generovat tři signály PWM, je nutné tyto bloky vytvořit právě tři. Counter je však vyvořen jen jeden a je všemi bloky sdílen.  


\sec Dekódování IRC

Dekódování výstupů IRC senzoru je realizováno v komponentě "qcounter.vhdl". 

Vstupem komponenty jsou signály "irc_A" a "irc_B". Podle pořadí, v jakém dochází ke změnám těchto dvou signálů,je možné odvodit, v jakém směru se motor otáčí.

Signály "irc_A" a "irc_B" jsou nejprve synchronně vzorkovány DFF obvodem spouštěným hlavní hodinovou frekvencí "gpio_clk". Aby došlo ke ztrátě informace o poloze, muselo by tak dojít ke dvěma změnám jednoho signálu během jednoho hodinového cyklu. Motor by se tedy musel za tuto dobu pootočit o tři body IRC. To by při frekvenci "gpio_clk" 50MHz znamenalo frekvenci otáčení  nejméně 75KHz, tedy 4500 tisíc otáček za minutu, což se nepředpokládá.

Nejmenší dva bity výsledné pozice jsou pak vytořeny čistě kombinačně z navzorkovaných "irc_A" a "irc_B":
  \ttline-1   
  \begtt
  qcount(0) <= a xor b;
  qcount(1) <= b;\endtt
 
Přechod do vyššího řádu je pak řešen pomocí 30-bitového čítače "count":
\begtt
	if (a_prev = '0') and (b_prev = '1') and (a = '0') and (b = '0') then 
			count <= count_prev + 1;
		elsif (a_prev = '0') and (b_prev = '0') and (a = '0') and (b = '1') then 
			count <= count_prev - 1;
		else
			count <= count_prev;
		end if;\endtt
Veličiny s příponou "_prev" jsou pak veličiny zpožděné o jeden hodinový cyklus oproti těm bez přípony. Čítač je nakonec do výstupu komponenty promítnut následovně:
\begtt
  qcount(31 downto 2) <= count;\endtt


\sec Implementace požadavků RT patche v aplikaci 

\secc RT thread

\sec Elektronická komutace

\secc Komutace s využitím pouze Hallových sond

\secc Komutace pomocí přičítání 120 stupňů

 \chap Závěr
 

 



 \bibchap
 \usebib/c (simple) mybase

\app Zadání

 \sec pokyny
 
Na platformě procesorové desky Raspberry Pi implementujte systém pro řízení bezkartáčových (BLDC/PMSM) motorů.

1. Pro komunikaci procesorového systému s výkonovým hardwarem realizovaným s využitím programovatelného obvodu (FPGA) vyberte vhodný protokol a periferii.

2. Pro vybraný způsob komunikace navrhněte ovladač na straně jádra Linux a obvodový návrh ve VHDL na straně FPGA.

3. Integrujte bloky pro snímání polohy, řízení výkonových stupňů a měření proudu do FPGA návrhu.

4. S využitím navržených periferií realizujte řízení bezkartáčového motoru.

5. Vyžaduje se podrobná technická dokumentace včetně přípravy podkladů pro prezentaci včetně videozáznamu.

\app Zkratky\par \makeglos %vlozi novou prilohu

\label[priloha_motor]
\app Dokumentace výkonového stupně Rpi-Mi-1, PiKRON 2014


\topinsert 
\picw=16cm \cinspic 3p-motor-driver-1-sch.pdf
\caption/f Schéma driveru motoru
\endinsert 

\topinsert 
\picw=15.8cm \cinspic rpi-mi-1-sch.pdf
\caption/f Schéma propojení RPi a FPGA obvodu
\endinsert  
 
 \bye
  
 